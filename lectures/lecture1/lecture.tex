\documentclass[t]{beamer}
\title{Building Recommender Systems}
\subtitle{Lecture 1}
\author{Boris Shminke}
\institute{Université Côte d'Azur, CNRS, LJAD, France}
\date{06/10/2022}
\AtBeginSection[]{
  \begin{frame}
    \frametitle{Outline}
    \tableofcontents[currentsection]
  \end{frame}
}
\begin{document}
\begin{frame}
  \titlepage  
\end{frame}  
\begin{frame}
  \frametitle{Outline}
  \tableofcontents
\end{frame}
\section{Administrative Questions}
\begin{frame}
  \frametitle{Why I'm teaching this course}
  \begin{itemize}
  \item building recommender systems (RS) from 2015
  \item RS development team leader in ivi.ru for three years
  \item RS R\&D team leader in Sberbank AI lab for two years
  \end{itemize}
\end{frame}
\begin{frame}
  \frametitle{The course structure}
  \begin{itemize}
  \item 8 three-hour classes
  \item one course project
  \item starts today, ends in January
  \end{itemize}
\end{frame}
\begin{frame}
  \frametitle{A class structure}
  \begin{itemize}
  \item $\sim1.5$ hours lecture
  \item $\sim15$ minutes break
  \item $\sim1.5$ hours practical session
  \end{itemize}
\end{frame}
\begin{frame}
  \frametitle{What's on Moodle}
  \begin{itemize}
  \item lecture slide decks
  \item practical tasks and solutions
  \item Zoom links (if you can't come to the class)
  \item \href{https://lms.univ-cotedazur.fr/2022/mod/forum/view.php?id=220784}{forum}
  \item course \href{https://lms.univ-cotedazur.fr/2022/mod/assign/view.php?id=234467}{project} and grading
  \end{itemize}
\end{frame}
\begin{frame}
  \frametitle{Grading details}
  \begin{itemize}
  \item only course project is important
  \item no obligatory homework
  \item no exams
  \end{itemize}
\end{frame}
\begin{frame}
  \frametitle{How to contact me}
  \begin{itemize}
  \item boris.shminke@univ-cotedazur.fr
  \item through Moodle \href{https://lms.univ-cotedazur.fr/2022/mod/forum/view.php?id=220784}{forum}
  \end{itemize}
\end{frame}
\begin{frame}
  \frametitle{Peut-on parler français dans ce cours ?}
  \begin{itemize}
  \item Boris only learns French, his main working language is English, his mother tongue is Russian
  \item Boris is not aware of any good books or educational resources on RS in French (if you find any, please share through the forum!)
  \item Vous êtes bienvenus de rédiger votre projet et les devoirs en français !
  \end{itemize}
\end{frame}
\begin{frame}
  \frametitle{Parler anglais, c'est difficile pour moi...}
  \begin{itemize}
  \item pas de souci, vous pouvez poser des questions (y compris dans le forum !) en français
  \item Boris peut souvent comprendre vos questions
  \item mais d'habitude il ne peut pas expliquer les choses en français, car il ne connaît pas les mots (ou la terminologie n'existe pas encore en français)
  \item essayer de lire, écouter et comprendre l'anglais, mais garder français pour parler si ça marche pour vous
  \end{itemize}
\end{frame}
\section{Introduction to Recommender Systems}
\begin{frame}
  \frametitle{Recommender systems are everywhere}
  \begin{itemize}
  \item videos on YouTube
  \item tracks on Spotify
  \item posts on Facebook
  \item books on Amazon
  \item hotels on Booking
  \item jobs on LinkedIn
  \end{itemize}
\end{frame}
\begin{frame}
  \frametitle{Well-established field of research}
  \begin{itemize}
  \item 1990s --- first papers on RS published
  \item 1997 --- GroupLens (RS for Usenet News) launched
  \item 2006–2009 --- Nextflix Prize
  \item 2007 --- first ACM RecSys Conference
  \item 2010 --- first RecSys Challenge
  \item 2010 --- Recommender Systems Handbook first edition
  \end{itemize}
\end{frame}
\begin{frame}
  \frametitle{Recommended books}
  \begin{itemize}
  \item \href{https://doi.org/10.1007/978-1-0716-2197-4}{Recommender Systems Handbook} --- a new (2022) edition of a fundamental book about everything RS
  \item \href{https://doi.org/10.1017/CBO9780511763113}{Recommender Systems: An Introduction} --- a slightly aged (2010) book which nonetheless discusses practical aspects of RS relevant until today
  \item \href{https://www.amazon.com/Collaborative-Recommendations-Algorithms-Challenges-Applications/dp/9813275340}{Collaborative Recommendations: Algorithms, Practical Challenges and Applications} --- a more theoretical treaty on arguably the most successful RS paradigm
  \end{itemize}
\end{frame}
\begin{frame}
  \frametitle{Fundamental Terms}
  \begin{itemize}
  \item `items' --- something being recommended
  \item `user' --- can interact with items (recommended and not)
  \item `rating' --- quantified feedback after user-item interaction
  \end{itemize}
\end{frame}
\begin{frame}
  \frametitle{User-Item Ratings Matrix}
\begin{center}
\begin{tabular}{ |c|c|c|c|c| }
 \hline
 & $item_1$ & $item_2$ & $item_3$ & $item_4$\\
 \hline
 $user_1$ & $1$ & & $5$ & $-3$\\
 \hline
 $user_2$ & & $0$ & $1$ &\\
 \hline
 $user_3$ & $-2$ & $1.5$ & &\\
 \hline
 $user_4$ & & & $-2.3$ & $0$\\
 \hline
\end{tabular}
\end{center}
\end{frame}
\begin{frame}
  \frametitle{Recommender system cycle}
  \begin{itemize}
  \item user connects to the system
  \item systems provides a structured set of items
  \item user interacts with items
  \item feedback is collected
  \item system learns
  \item (another) user connects
  \item ...
  \end{itemize}
\end{frame}
\begin{frame}
  \frametitle{What an RS returns to a user?}
  \begin{itemize}
  \item one best item (ad at some spot, next video auto-play)
  \item a handful of items (you can also like, special offer)
  \item ranked lists (romantic comedies personalized sorting)
  \item list of lists (YouTube/Netflix main page)
  \item a tree (a menu of lists of lists of lists of lists...)
  \item set of items organized in any structure
  \end{itemize}
\end{frame}
\begin{frame}
  \frametitle{Explicit feedback}
  \begin{itemize}
  \item review
  \item questionnaire
  \item stars
  \item like/dislike
  \end{itemize}
\end{frame}
\begin{frame}
  \frametitle{Explicit feedback}
  \begin{itemize}
  \item complex (self-contradictory sometimes)
  \item hard to collect (and thus scarce)
  \item biased
  \item all people are different
  \item everybody lies
  \end{itemize}
\end{frame}
\begin{frame}
  \frametitle{Implicit feedback}
  \begin{itemize}
  \item number of times a user listened to a soundtrack
  \item number of episodes of the show a user watched
  \item sum of money a user spent on a particular brand
  \item time a user spent reading articles on that topic
  \item any numeric measurement of interaction
  \end{itemize}
\end{frame}
\begin{frame}
  \frametitle{Implicit feedback}
  \begin{itemize}
  \item simple (one number)
  \item cheap and abundant
  \item biased
  \item hard to interpret
  \item can be gamed on purpose
  \end{itemize}
\end{frame}
\begin{frame}
  \frametitle{More terminology}
  \begin{itemize}
  \item item2user (aka personal; most focus here)
  \item item2item (users who bought this also bought)
  \item user2user (friend suggestion, prediction of missing edges in a graph)
  \end{itemize}
\end{frame}
\begin{frame}
  \frametitle{Recommender system vs Search}
  \begin{itemize}
  \item user is an (ever changing) query
  \item there is no ideal ranking
  \item recommending top-down list is not always the best idea
  \end{itemize}
\end{frame}
\begin{frame}
  \frametitle{Recommender system vs Advertisement}
  \begin{itemize}
  \item users asks for recommendations
  \item the systems doesn’t need to promote anything
  \item retention is more important than selling now
  \end{itemize}
\end{frame}
\begin{frame}
  \frametitle{Non-example: a few items}
  \begin{itemize}
  \item banking cards, telecom tariffs, master programs
  \item show everything or
  \item build a separate classification model for each item
  \end{itemize}
\end{frame}
\begin{frame}
  \frametitle{Non-example: repeating the action}
  \begin{itemize}
  \item even if we have gazillions of items
  \item every user interacts only with a very small number of them
  \item recommending what track to relisten, book to reread, etc
  \item is usually solved as a classification problem
  \end{itemize}
\end{frame}
\begin{frame}
  \frametitle{Non-example: items are too different}
  \begin{itemize}
  \item sums of withdrawal in an ATM, plane tickets for a holiday
  \item Q\&A system
  \item classification can work as well
  \end{itemize}
\end{frame}
\begin{frame}
  \frametitle{Non-example: item's value is objective}
  \begin{itemize}
  \item financial market products
  \item value is universal, no need for personalisation
  \item predicting true value is more important
  \end{itemize}
\end{frame}
\begin{frame}
  \frametitle{Rules of thumb for an RS application}
  \begin{itemize}
  \item user can’t observe all items during one séance (100-1000 or more)
  \item `good' items are nearly interchangeable for user
  \item user can't sometimes discern between `good' and `bad'
  \end{itemize}
\end{frame}
\section{RS Validation}
\begin{frame}
  \frametitle{Accuracy metrics}
  \begin{itemize}
  \item how often a user interacts with an item we recommended
  \item how many interactions are generated from recommendations
  \item position is important (ranking metrics are welcome)
  \item easy to quantify
  \end{itemize}
\end{frame}
\begin{frame}
  \frametitle{Diversity}
  \begin{itemize}
  \item filter bubbles are bad
  \item demonstration of item catalogue diversity is crucial for user retention
  \item harder to quantify than accuracy
  \end{itemize}
\end{frame}
\begin{frame}
  \frametitle{Novelty}
  \begin{itemize}
  \item recommend something which a user was not aware of
  \item but what looks for her as a plausible choice
  \item quite hard to quantify (unexpectedness paper)
  \end{itemize}
\end{frame}
\begin{frame}
  \frametitle{Serendipity}
  \begin{itemize}
  \item recommends something which doesn't look like a good option
  \item but user gives a try and loves it
  \item nearly impossible to quantify
  \item but probably the most important quality of RS
  \end{itemize}
\end{frame}
\begin{frame}
  \frametitle{So, how to start}
  \begin{itemize}
  \item popular items are hard to beat in accuracy
  \item diversity can be added by balancing genres etc
  \item difference between a basic RS and advanced one for most users are small on average
  \end{itemize}
\end{frame}
\begin{frame}
  \frametitle{How exactly to calculate metrics?}
  \begin{itemize}
  \item rs\_metrics: \url{https://darel13712.github.io/rs_metrics/}
  \item Python package, easy to install and use
  \item ten popular metrics implemented
  \end{itemize}
\end{frame}
\end{document}
